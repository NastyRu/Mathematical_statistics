\documentclass[a4paper,14pt]{extreport} % формат документа

\usepackage{amsmath}
\usepackage{cmap} % поиск в ПДФ
\usepackage[T2A]{fontenc} % кодировка
\usepackage[utf8]{inputenc} % кодировка исходного текста
\usepackage[english,russian]{babel} % локализация и переносы
\usepackage[left = 2cm, right = 1cm, top = 2cm, bottom = 2 cm]{geometry} % поля
\usepackage{listings}
\usepackage{graphicx} % для вставки рисунков
\usepackage{amsmath}
\usepackage{amssymb}
\usepackage{float}
\usepackage{multirow}
\graphicspath{{pictures/}}
\usepackage{amsfonts}
\usepackage{breqn}
\DeclareGraphicsExtensions{.pdf,.png,.jpg}
\newcommand{\anonsection}[1]{\section*{#1}\addcontentsline{toc}{section}{#1}}

\renewcommand{\thesection}{\arabic{section}}

\begin{document}
\begin{titlepage}

    \begin{table}[H]
        \centering
        \footnotesize
        \begin{tabular}{cc}
            \multirow{8}{*}{\includegraphics[scale=0.35]{bmstu.jpg}}
            & \\
            & \\
            & \textbf{Министерство науки и высшего образования Российской Федерации} \\
            & \textbf{Федеральное государственное бюджетное образовательное учреждение} \\
            & \textbf{высшего образования} \\
            & \textbf{<<Московский государственный технический} \\
            & \textbf{университет имени Н.Э. Баумана>>} \\
            & \textbf{(МГТУ им. Н.Э. Баумана)} \\
        \end{tabular}
    \end{table}

    \vspace{-2.5cm}

    \begin{flushleft}
        \rule[-1cm]{\textwidth}{3pt}
        \rule{\textwidth}{1pt}
    \end{flushleft}

    \begin{flushleft}
        \small
        ФАКУЛЬТЕТ
        \underline{<<Информатика и системы управления>>\ \ \ \ \ \ \ 
        \ \ \ \ \ \ \ \ \ \ \ \ \ \ \ \ \ \ \ \ \ \ \ \ \ \ \ \ \ \ \ 
    \ \ \ \ \ \ \ \ \ \ \ \ \ \ \ } \\
        КАФЕДРА
        \underline{<<Программное обеспечение ЭВМ и
        информационные технологии>>
        \ \ \ \ \ \ \ \ \ \ \ \ \ \ \ \ \ \ \ \ }
    \end{flushleft}

    \vspace{2cm}

    \begin{center}
        \textbf{Домашняя работа № 2} \\
        \vspace{0.5cm}
    \end{center}

    \vspace{5cm}

    \begin{flushleft}
        \begin{tabular}{ll}
            \textbf{Дисциплина} & Математическая статистика.  \\
            \textbf{Студент} & Сиденко А.Г. \\
            \textbf{Группа} & ИУ7-63Б \\
            \textbf{Вариант} & 22 \\
            \textbf{Оценка (баллы)} & \\
            \textbf{Преподаватель} & Власов П.А.   \\
        \end{tabular}
    \end{flushleft}

    \vspace{4cm}

   \begin{center}
        Москва, 2020 г.
    \end{center}

\end{titlepage}

\section{Задача 1 (проверка параметрических гипотез)}

\hfill

\textbf{Условие}

Давление в камере измеряется двумя манометрами. Для сравнения точности этих приборов через некоторые промежутки времени были $n = 10$ раз синхронно сняты их показания, в результате чего получены значения (в единицах шкалы приборов) $\overline x_n = 1573$, $S^2(\vec x_n) = 0.72$ (для первого прибора) и $\overline y_n = 1671$, $S^2(\vec y_n) = 0.15$ (для второго прибора). Считая распределение ошибок нормальным, с использованием одностороннего критерия при уровне значимости $\alpha = 0.01$ проверить гипотезу о равенстве дисперсий.

\hfill

\textbf{Решение}

\begin{enumerate}

\item Введем нулевую гипотезу. 

$H_0=\{\text{Дисперсии равны: } \sigma_1=\sigma_2\}$

\item Конкурирующая гипотеза. 

$H_1=\{\sigma_1>\sigma_2\}$

\item Используем статистику. 

$T(\vec x_n,\vec y_n)\sim F(n-1, n-1)$

\item Построим критическое множество. 

$W=\{(\vec x, \vec y):T(\vec x_n,\vec y_n)\ge F_{1-\alpha}(n-1, n-1)\}$, где

$F_{1-\alpha}$ -- квантиль распределения Фишера. 

\item Вычислим статистику. 

$$T(\vec x_n,\vec y_n)=\frac{\max\{S^2(\vec x_n), S^2(\vec y_n)\}}{\min \{S^2(\vec x_n), S^2(\vec y_n)\}}=\frac{S^2(\vec x_n)}{S^2(\vec y_n)}=\frac{0.72}{0.15}=4.8$$

\item Значение квантили узнаем с помощью функции finv в Matlab:

$F_{1-\alpha}(9,9)=F_{0.99}=5.35$

\item Вывод

$4.8 \ngeq 5.35~ \Rightarrow ~ (\vec x, \vec y) \notin W~ \Rightarrow~$ отвергаем гипотезу $H_1$, принимаем $H_0$. 

\end{enumerate}

\hfill

\textbf{Ответ:}

При уровне значимости $\alpha=0.01$ дисперсии параметров манометров равны. 

\end{document}